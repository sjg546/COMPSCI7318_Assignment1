\documentclass[10pt,twocolumn,letterpaper]{article}

\usepackage{cvpr}
\usepackage{times}
\usepackage{epsfig}
\usepackage{graphicx}
\usepackage{amsmath}
\usepackage{amssymb}
\usepackage{mathtools}
% Include other packages here, before hyperref.

% If you comment hyperref and then uncomment it, you should delete
% egpaper.aux before re-running latex.  (Or just hit 'q' on the first latex
% run, let it finish, and you should be clear).
\usepackage[breaklinks=true,bookmarks=false]{hyperref}

\cvprfinalcopy % *** Uncomment this line for the final submission

\def\cvprPaperID{****} % *** Enter the CVPR Paper ID here
\def\httilde{\mbox{\tt\raisebox{-.5ex}{\symbol{126}}}}

% Pages are numbered in submission mode, and unnumbered in camera-ready
%\ifcvprfinal\pagestyle{empty}\fi
\setcounter{page}{4321}
\begin{document}

%%%%%%%%% TITLE
\title{Perceptron}

\author{Sam Gilbert\\
   a1737770\\}
\maketitle
%\thispagestyle{empty}

%%%%%%%%% ABSTRACT
\begin{abstract}
   The perceptron algorithm is one of the simplest machine learning algorithms. It is a
   form of supervised learning that
\end{abstract}

%%%%%%%%% BODY TEXT
\section{Introduction}

Perceptron is one of the simplest machine learning algorithms. Perceptron is a supervised learning
algorithm that takes a combination of parameters to make a binary classification.
Perceptron works by iteratively calculating a set of weights based on the dimension size
of the features in the dataset, $D$. The output of which is a line or $D$ dimension plane
that separates the observations into two categories. Perceptron calculates the updated
weights iteratively and in an ideal world will converge to a $D$ dimension vector of values
that separates all values into their correct groups. Where the data is not able to be
separated cleanly into two groups the Perceptron algorithm will never converge. This requires
a max iteration value to be set such that the problem is bounded. If a max iteration value
is not set the Perceptron algorithm can keep moving back and forth trying to cleanly separate
the training observations into the correct groups. Determining the optimal max iterations
for the dataset will be investigated in the method section of the report.

In this project the Perceptron algorithm was implemented on the Pima Indians Diabetes Database
open source dataset \cite{kagglePimaIndians}. This dataset contains a set of eight medical
predictor variables, the number of pregnancies the patient has had, their BMI, insulin level,
plasma glucose concentration, blood pressure, skin fold thickness, age and their diabetes
pedigree. The dataset also contains an outcome variable which is either the patient has
diabetes or they don't have diabetes. The predictor variables are each one of the $D$
dimensions.
%% talk about activation function

\section{Method}

As outlined in the introduction section the goal of the Perceptron algorithm is to separate
the entries in the dataset into one of two binary groups. The data is read into a python array
of dictionaries with the features being stored in a $D$ dimensional array and the outcome
stored as a signed integer. The data is then split into a training and testing set using the
scikit learn train test function. An 80-20 training testing split was used \cite{gholamy_why_nodate}.

The Perceptron algorithm takes an input layer which is the set of features in each observation, let
the vector of features for observation $i$ be defined as $x_i$. Let the collection of these observations
be defined as $X$. Each of these observations have an outcome variable which is whether the patient has
diabetes or not. Let the outcome for the $i_{th}$ observation be defined as $y_i$. This outcome variable
is either $+1$ if the patient has diabetes or $-1$ if the patient doesn't have diabetes. Let the collection
of all the outcomes for each observation be defined as $Y$. The Perceptron algorithm starts by creating a $D$
dimension vector of weights, defined as $W$. For this implementation the initial $W$ is set as a $D$ dimension
zero vector. A bias value is also required which is defined as $b$ initially set to $0$. This bias shifts the
intercept of the dimensions linear separator away from $0$ and is required as it may not be possible to
split the data points if this doesn't occur.

The Perceptron algorithm works by calculating the dot product of $x_i$ and $W$ and adding the bias, $b$.
The sign function is then passed the result and this is our activation function corresponding to the possible
values of the outcome variable mentioned above. The sign function is defined as
\[
   sign(x)= \begin{cases*}
      -1 & if x $<$ 0 \\
      0  & if x = 0   \\
      +1 & if x $>$ 0 \\
   \end{cases*}
\]

To determine if the perceptron has correctly classified the observation, the result of the sign function
corresponds to the outcome variable possibilities of whether a patient has diabetes or not.
\begin{equation}
   x_i \cdot W
\end{equation}
for a single observation.



   %-------------------------------------------------------------------------

   {\small
      \bibliographystyle{ieee_fullname}
      \bibliography{egbib}
   }

\end{document}
